\documentclass[12pt,oneside,openary]{article}
\usepackage{graphicx}                                                                                                                                          
%\usepackage[dvipdfmx]{graphicx}
%\usepackage{enumerate}
%\usepackage{mediabb}                                                                                                                                            
%\usepackage{amssymb}
%\usepackage{epstopdf}
%\usepackage{ascmac}
%\usepackage{float}
%\usepackage{breqn}
%\usepackage{comment}
%\usepackage{hyperref}
%\usepackage{framed}
%\usepackage{color}
\usepackage{amsmath}
\usepackage{fullpage}
%\usepackage{here}
%\usepackage{fancyhdr}
%\usepackage{lineno}
%\pagestyle{fancyplain}

\begin{document}
\title{Derivation of Angle Dependence in Boost and Rotation}
\author{Theo Koblesky}
\maketitle
%\email{s1420259@u.tsukuba.ac.jp}                                                                                                                                
%\linenumbers

%\begin{equation}
For this derivation, I am using the natural units system (c = 1).
\begin{equation}
p_{blue} = (\sqrt{100^2+m_{proton}^2},100\sin(\theta_{blue}),0,100\cos(\theta_{blue})) 
\end{equation}
\begin{equation}
p_{yellow} = (\sqrt{100^2+m_{proton}^2},100Sin(\theta_{yellow}+\pi),0,100Cos(\theta_{yellow}+\pi)) 
\end{equation}
Assuming $\theta_{blue} and \theta_{yellow}$ are small (of the order of 10s of mRad), we can reduce these components to
\begin{equation}
p_{blue} = (\sqrt{100^2+m_{proton}^2},100\theta_{blue},0,100)) 
\end{equation}
\begin{equation}
p_{yellow} = (\sqrt{100^2+m_{proton}^2},-100\theta_{yellow},0,-100) 
\end{equation}
\begin{equation}
p_{CMS} = (p_{blue}+p_{yellow})
\end{equation}
\begin{equation}
v = p/E
\end{equation}
boost velocity vector:
\begin{equation}
v_{B} = (p_{xCMS},0,0)/E_{CMS}
\end{equation}
Small angle approximation and small mass limit:
\begin{equation}
v_{Bx}  = \frac{100(\theta_{blue}-\theta_{yellow})}{2\sqrt{100^2+m_{proton}^2}}\approx 0.5(\theta_{blue}-\theta_{yellow})
\end{equation}
\begin{equation}
v_{Bz} = \frac{100(Cos(\theta_{blue})+Cos(\theta_{yellow}+\pi))}{2\sqrt{100^2+m_{proton}^2}}\approx 0.25(\theta_{yellow}^2-\theta_{blue}^2)
\end{equation}
Boost Matrix (for x and z):
\[
B=
\begin{bmatrix}
    \gamma & -\gamma v_{x} & 0  &-\gamma v_{z} \\
    -\gamma v_{Bx} & 1+\frac{(\gamma-1)v_{Bx}^2}{v^2} & 0  & \frac{(\gamma-1)v_{Bx}v_{Bz}}{v^2} \\
    0 & 0 & 1 & 0 \\
    -\gamma v_{Bz} & \frac{(\gamma-1)v_{Bx}v_{Bz}}{v^2} & 0  & 1+\frac{(\gamma-1)v_{Bz}^2}{v^2}
\end{bmatrix}
\]
For a given 4 momentum vector of a single particle, we have
\begin{equation}
p = (E,p_x,p_y,p_z).
\end{equation}
Boosted momentum in the limit that $\gamma$ = 1:
\begin{equation}
p_{x}\prime=-\gamma v_{Bx}E+p_x(1+\frac{(\gamma-1)v_{Bx}^2}{v^2})+p_z(\frac{(\gamma-1)v_{Bx}v_{Bz}}{v^2})\approx -v_{Bx}E+p_x
\end{equation}
\begin{equation}
p_{z}\prime=-\gamma v_{Bz}E+p_z(1+\frac{(\gamma-1)v_{Bz}^2}{v^2})+p_x(\frac{(\gamma-1)v_{Bx}v_{Bz}}{v^2})\approx -v_{Bz}E+p_z
\end{equation}
Now to calculate the rotation angle after the boost (assuming $p_z\prime$ is much greater than $p_x\prime$):
\begin{equation}
\theta_{xz}=\arctan(\frac{p_{Bluex}\prime}{p_{Bluez}\prime})\approx \frac{p_{Bluex}\prime}{p_{Bluez}\prime}
\end{equation}
Rotation matrix:
\[
R=
\begin{pmatrix}
    Cos(\theta_{xz})  & -Sin(\theta_{xz}) \\
     Sin(\theta_{xz})  & Cos(\theta_{xz})
\end{pmatrix}
\]
\begin{equation}
p_{x}\prime\prime = p_{x}\prime Cos(\theta_{xz})- p_{z}\prime Sin(\theta_{xz})\approx p_{x}\prime - p_z\prime \theta_{xz}
\end{equation}
\begin{equation}
p_{z}\prime\prime = p_{x}\prime Sin(\theta_{xz})+ p_{z}\prime Cos(\theta_{xz})\approx p_{x}\prime \theta_{xz} + p_z\prime 
\end{equation}
Rewriting $p_x\prime$ in terms of $p_x$, we get
\begin{equation}
p_{x}\prime\prime = p_{x} - E v_{Bx} - (p_z - E v_{Bz}) \frac{p_{Bluex} - E_{Blue} v_{Bx}}{p_{Bluez} - E_{Blue} v_{Bz}}.
\end{equation}
Dropping higher order velocity terms, we get
\begin{equation}
p_{x}\prime\prime = p_{x} - E v_{Bx} - p_z  \frac{p^{Blue}_{x} - E_{Blue} v_{Bx}}{p_{Bluez}},
\end{equation}
and then rewriting the boost velocity in terms of the beam angle, we obtain
\begin{equation}
p_{x}\prime\prime = p_{x} - 0.5 E (\theta_{Blue}-\theta_{Yellow}) - \left(\frac{p_z p_{Bluex}}{p_{Bluez}}\right)+\frac{0.5 p_{z}E_{Blue}(\theta_{Blue}-\theta_{Yellow})}{p_{Bluez}}.
\end{equation}
Using the approximations $p_{Bluex}/p_{Bluez} \approx \theta_{blue}$ and $E_{Blue}/p_{Bluez} \approx 1$,
we obtain
\begin{equation}
p_{x}\prime\prime = p_{x} -  0.5E(\theta_{Blue}-\theta_{Yellow}) - (p_z \theta_{Blue})+0.5 p_{z} (\theta_{Blue}-\theta_{Yellow}).
\end{equation}
If we define $F_{Blue} = 0.5(E+p_z)$
and 
$F_{Yellow} = 0.5(E-p_z)$,
we obtain
\begin{equation}
p_{x}\prime\prime = p_{x} -\theta_{Blue}F_{Blue} - \theta_{Yellow}F_{Yellow}.
\end{equation}
For a charged pion at eta = -3.5 with pt = 250 MeV, E = 4.14 GeV and pz = -4.13 GeV/c, we get the numerical results
$F_{Blue} = 0.00495$
and 
$F_{Yellow} = -4.14$.

\begin{figure}
\begin{center}
\includegraphics[width=0.5\linewidth]{name.pdf}
\caption{Diagram showing the relevant beam angles.}
\label{fig:diagram1}
\end{center}
\end{figure}

%We can set up the equivalent rotation using the yellow beam:
%\begin{equation}
%\theta_{xz}=ATan(\frac{p_{Yellowx}\prime}{p_{Yellowz}\prime})\approx \frac{p_{Yellowx}\prime}{p_{Yellowz}\prime}
%\end{equation}
%\[
%R=
%\begin{pmatrix}
 %   Cos(\theta_{xz})  & Sin(\theta_{xz}) \\
  %   -Sin(\theta_{xz})  & Cos(\theta_{xz})
%\end{pmatrix}
%\]
%\begin{equation}
%p_{x}\prime\prime = p_{x}\prime Cos(\theta_{xz}) +p_{z}\prime Sin(\theta_{xz})\approx p_{x}\prime + p_z\prime \theta_{xz}
%\end{equation}
%\begin{equation}
%p_{z}\prime\prime = -p_{x}\prime Sin(\theta_{xz})+ p_{z}\prime Cos(\theta_{xz})\approx -p_{x}\prime \theta_{xz} + p_z\prime 
%\end{equation}

\end{document}

%%
%% End of file `nppp-template.tex'. 

% LocalWords:  NN
